% GA4GH File Encryption Standard specification
% When editing, please start all sentences on a new line; and try to avoid source lines longer than 120 characters.
\documentclass[10pt]{article}
\usepackage[margin=1in]{geometry}
\usepackage[pdfborder={0 0 0},hyperfootnotes=false]{hyperref}
\usepackage{parskip}
\usepackage{draftwatermark}
\newcommand{\mailtourl}[1]{\href{mailto:#1}{\tt #1}}
\newcommand{\tagvalue}[1]{\tt #1}
\newcommand{\tagregex}[1]{\tt #1}

\SetWatermarkAngle{315}
\SetWatermarkText{PRELIMINARY}
\SetWatermarkScale{4}

\begin{document}

% GA4GH File Encryption Standard specification
% When editing, please start all sentences on a new line; and try to avoid source lines longer than 120 characters.
\documentclass[10pt]{article}
\usepackage[margin=1in]{geometry}
\usepackage[pdfborder={0 0 0},hyperfootnotes=false]{hyperref}
\usepackage{parskip}
\newcommand{\mailtourl}[1]{\href{mailto:#1}{\tt #1}}
\newcommand{\tagvalue}[1]{\tt #1}
\newcommand{\tagregex}[1]{\tt #1}
\begin{document}

% GA4GH File Encryption Standard specification
% When editing, please start all sentences on a new line; and try to avoid source lines longer than 120 characters.
\documentclass[10pt]{article}
\usepackage[margin=1in]{geometry}
\usepackage[pdfborder={0 0 0},hyperfootnotes=false]{hyperref}
\usepackage{parskip}
\newcommand{\mailtourl}[1]{\href{mailto:#1}{\tt #1}}
\newcommand{\tagvalue}[1]{\tt #1}
\newcommand{\tagregex}[1]{\tt #1}
\begin{document}

% GA4GH File Encryption Standard specification
% When editing, please start all sentences on a new line; and try to avoid source lines longer than 120 characters.
\documentclass[10pt]{article}
\usepackage[margin=1in]{geometry}
\usepackage[pdfborder={0 0 0},hyperfootnotes=false]{hyperref}
\usepackage{parskip}
\newcommand{\mailtourl}[1]{\href{mailto:#1}{\tt #1}}
\newcommand{\tagvalue}[1]{\tt #1}
\newcommand{\tagregex}[1]{\tt #1}
\begin{document}

\input{crypt4gh.ver}
\title{GA4GH File Encryption Standard}
\date{\headdate}
\maketitle
\begin{quote}\small
The master version of this document can be found at
\url{https://github.com/samtools/hts-specs}.\\
This printing is version~\commitdesc\ from that repository,
last modified on the date shown above.
\end{quote}
\vspace*{1em}

\begin{abstract}
This document describes the format for Global Alliance for Genomics and Health (GA4GH) encrypted and authenticated
files.
Encryption helps to prevent accidental disclosure of confidential information.
Allowing programs to directly read and write data in an encrypted format reduces the chance of such disclosure.
The format described here can be used to encrypt any underlying file format.
It also allows for seeking on the encrypted data.
In particular indexes on the plain text version can also be used on the encrypted file without modification.
\end{abstract}
\newpage
\tableofcontents
\newpage
\section{Introduction}
\subsection{Purpose}
By its nature, genomic data can include information of a confidential nature about the health of individuals.
It is important that such information is not accidentally disclosed.
One part of the defence against such disclosure is to, as much as possible, keep the data in an encrypted format.

This document describes a file format that can be used to store data in an encrypted and authenticated state.
Existing applications can, with minimal modification, read and write data in the encrypted format.
The choice of encryption also allows the encrypted data to be read starting from any location, facilitating indexed
access to files.

\subsection{Requirements}
The key words ``MUST'', ``MUST NOT'', ``REQUIRED'', ``SHALL'', ``SHALL NOT'', ``SHOULD'', ``SHOULD NOT'',
``RECOMMENDED'', ``MAY'', and ``OPTIONAL'' in this document are to be interpreted as described in \cite{RFC2119}.

\subsection{Terminology}
\begin{description}
\item[Elliptic-curve cryptography (ECC)]
An approach to public-key cryptography based on the algebraic structure of elliptic curves over finite fields.
\item[Curve25519]
A widely used FIPS-140 approved ECC algorithm not encumbered by any patents \cite{RFC7748}. 
\item[Ed25519]
An Edwards-curve Digital Signature Algorithm (EdDSA) over Curve25519 \cite{RFC8032}.
\item[ChaCha20-ietf-Poly1305]
ChaCha20 is a symmetric stream cipher built on a pseudo-random function that gives the advantage that one can 
efficiently seek to any position in the key stream in constant time.
It is not patented.
Poly1305 is a cryptographic message authentication code (MAC).
It can be used to verify the data integrity and the authenticity of a message \cite{RFC8439}.

\item[cipher-text]~\\
The encrypted version of the data.

\item[plain-text]~\\
The unencrypted version of the data.
 
\end{description}

\section{Encrypted Representation Overview}
The encrypted file consists of four parts:

\begin{itemize}
\item An unencrypted header, containing a magic number, version number, length of public key 
(may be zero), public key, a header signature indicator, and the length of the encrypted header.

\item An encrypted header, which is encrypted using an asymmetric encryption algorithm.
It lists the encryption key and nonce needed to decrypt the encrypted data section.

\item A 64-byte Ed25519 signature validating the encrypted header.

\item The encrypted data.
This is the actual application data.
It is encrypted using a symmetric encryption algorithm as described in the encrypted header.
The data is encrypted in 64K segments with a 16 byte MAC appended at the end of each segment.
\end{itemize}

\section{Detailed Specification}
\subsection{Overall Conventions}
\subsubsection{Hexadecimal Numbers}
Hexadecimal values are written using the digits 0-9, and letters a-f for values 10-15.
Values are written with the most-significant digit on the left, and prefixed with "0x".

\subsubsection{Byte Ordering}
The basic data size is the byte (8 bits).
Multi-byte values need to be stored in a defined order.
This order will be one of the following:
\begin{description}
\item[Least-significant byte first (``little-endian'')]~\\
The value 1234 decimal (0x4d2) is stored as the byte stream 0xd2 0x04.

\item[Most-significant byte first (``big-endian'')]~\\
The value 1234 decimal (0x4d2) is stored as the byte stream 0x04 0xd2.
\end{description}

\subsubsection{Integer Types}
Integers can be either signed or unsigned.
Signed values are stored in two's complement form.

\subsubsection{Multi-byte Integer Types}
\begin{center}
\begin{tabular}{l l l l}
\hline
\textbf{Name} & \textbf{Byte Ordering} & \textbf{Integer Type} & \textbf{Size (bytes)} \\
\hline
byte & & unsigned & 1 \\
le\_int32 & little-endian & signed & 4 \\
le\_uint32 & little-endian & unsigned & 4 \\
le\_int64 & little-endian & signed & 8 \\
le\_uint64 & little-endian & unsigned & 8 \\
be\_uint128 & big-endian & unsigned & 16 \\
\end{tabular}
\end{center}

\subsubsection{Vectors}
A vector is a stream of elements of the same type (which may be a structure).
The number of items may be specified either as a constant value or in reference to a known integer value (for
example, a variable previously read from the file).

\begin{verbatim}
le_in32   num     // Number of v2 array elements
le_int32  v1[8]   // Eight four-byte little-endian integers
le_int64  v2[num] // 'num' eight-byte little-endian integers
\end{verbatim}

When vectors are serialized to a file, the elements are written with no padding between them.

\subsubsection{Structures}
Structure types may be defined for convenience.
The syntax for definition is similar to that of C.

\begin{verbatim}
struct demo {
  byte string[8];
  le_int32 number1;
  le_uint64 number2;
};
\end{verbatim}

When structures are serialized to a file, elements are written in the given order with no padding between them.
So the structure above would be written as twenty bytes - eight for the array `string', four for the integer `number1',
and eight for the integer `number2'.

\subsubsection{Enumerated Types}
Enumerated types may only take one of a given set of values.
The data type used to store the enumerated value is given in angle brackets after the type name.
Every element of an enumerated type must be assigned a value.
It is not valid to compare values between two enumerated types.

\begin{verbatim}
enum Animal<le_uint32> {
  cat    = 1;
  dog    = 2;
  rabbit = 3;
};
\end{verbatim}

\subsubsection{Variants}
Parts of structures may vary depending on information available at the time of decoding.
Which variant to use is selected by an enumerated type.
There must be a case for every possible enumerated value.
Cases have limited fall-through.
Consecutive cases with no fields in between all contain the same fields.

\begin{verbatim}
struct AnimalFeatures {
  select (enum Animal) {
    case cat:
    case dog:
      le_uint32 hairyness;
      le_uint32 whisker_length;

    case rabbit:
      le_uint32 ear_length;
  };
};
\end{verbatim}

For the `cat' and `dog' cases, `struct AnimalFeatures' is eight bytes long and contains two unsigned four-byte
little-endian values.
For the `rabbit' case it is four bytes long and contains a single four-byte little-endian value.

If the cases are different lengths (as above), then the size of the overall structure depends on the variant chosen.
There is NO padding to make the cases the same length unless it is explicitly defined.

\subsection{Unencrypted Header}

The file starts with an unencrypted header, with the following structure:

\begin{verbatim}
struct Unencrypted_header {
  byte      magic_number[8];
  le_uint32 version;
  le_uint32 public_key_length;
  byte      public_key[public_key_length];
  le_uint32 header_len;
};
\end{verbatim}

The magic\_number is the ASCII representation of the string ``crypt4gh''.

The version number is stored as a four-byte little-endian unsigned integer.
The current version number is 1.

The public key length may be zero.
If it is greater than zero then it is the length of the public key component of the private key used to sign the
header (this is not the same as the public key used to encrypt the header).

hdr\_len is the sum of the lengths of the unencrypted and encrypted headers.
If a public key is specified then it also includes the 64-byte header signature field appended to the encrypted header.
It is stored as a four-byte little-endian unsigned integer.
As it includes the unencrypted header, hdr\_len will always have a value of at least 16.

The current byte representation of the magic number and version is:
\begin{verbatim}
0x63 0x72 0x79 0x70 0x74 0x34 0x67 0x68 0x01 0x00 0x00 0x00
============= magic_number============= ===== version =====
\end{verbatim}

Possible header configurations:

If no public key is specified:
\begin{verbatim}
[magic number][version][0][header len][encrypted header]
\end{verbatim}

If a public key is specified:
\begin{verbatim}
[magic number][version][pub key len][public key][header len][encrypted header][header signature]
\end{verbatim}

\subsection{Encrypted Header}
\subsubsection{Encryption Method}
The encrypted header is encoded in the Curve25519 message format \cite{RFC7748}.

\subsubsection{Header Signature}
The header signature is generated using the Ed25519 algorithm \cite{RFC8032}.

\subsubsection{Plain-text Format}
The plain-text data encoded in the encrypted header has the following overall structure:

The `Encryption\_parameters' type is defined as:

\begin{verbatim}
enum Encryption_method<le_uint32> {
  chacha20_ietf_poly1305 = 0;
};

struct Encryption_parameters {
  byte	checksum[32];
  enum Encryption_method<le_uint32> method;

  select (method) {
    case chacha20_ietf_poly1305:
      byte       key[32];
      be_uint128 nonce;
  };
};
\end{verbatim}

`checksum' is an optional 32-byte SHA-256 checksum of the (unencrypted) data in the file.

`method' is an enumerated type that describes the type of encryption to be used.

`key' is a secret encryption key.
Treated as a concatenation of eight 32-bit little-endian integers.

`nonce' is a unique initialization vector.
In IETF version it is 12 bytes long and if combined with a 4 byte counter to produce a 16 byte initialization
vector (IV).

\subsection{Encrypted Data}
\subsubsection{ChaCha20 Mode Encryption}

Internally, ChaCha20 works like a block cipher used in counter mode.
It includes an internal block counter to avoid incrementing the nonce after each block.
The cipher-text is the message combined with the output of the stream cipher using the XOR operation, 
and doesn't include any authentication tag.
In IETF mode the nonce is 96 bits long and the counter is 32 bits long.

ChaCha20-Poly1305 uses ChaCha20 to encrypt a message and uses the Poly1305 algorithm to
generate a 16-byte MAC over the encrypted data, which is appended at the end of the 
cipher-text.
The MAC is generated for the whole cipher-text that is provided.
It is not possible to authenticate partially decrypted data.

While it is possible to decrypt individual blocks in ChaCha20, to authenticate the content 
it is necessary to decrypt the entire cipher-text,  with the authentication tag at the end. 
To retain streaming and random access capabilities it is necessary to ensure that segments 
of the data can be authenticated, without having to read and process the whole file.
In this format the cipher-text is divided into 64K segments.
A MAC is generated for each segment individually and appended to each 64K block of cipher-text.

A dual counter method is used, where the nonce is incremented for each 64K segment, and
the counter is incremented within each segment, producing a unique IV (nonce+counter) for 
each cipher block even if the cipher-text is very large. 

The cipher-text is decrypted by authenticating and decrypting the segment(s) enclosing
the requested byte range. 

Implementation details for ChaCha20-ietf-Poly1305are described in \cite{RFC8439}.

\subsubsection{Partial Blocks and Segments}
It is likely that the end of the file will not occur at an exact multiple of the cipher block or 
64K segment length.
In such cases, a block or segment must be padded to enable calculation of the authentication tag. 
Extra care must be taken to prevent a padding oracle attack (\url{https://en.wikipedia.org/wiki/Padding_oracle_attack}).

\section{Security Considerations}
\subsection{Threat Model}
This format is designed to protect files at rest and in transport from accidental disclosure.
Using authenticated encryption in individual segments mirrors solutions like Transport Layer 
Security (TLS) as described in \cite{RFC5246} and prevent undetected modification of segments.

\subsection{Selection of Key and Nonce}
The security of the format depends on attackers not being able to guess the encryption key (and to a lesser extent 
the nonce).
The encryption key MUST be generated using a cryptographically-secure pseudo-random number generator.
This makes the chance of guessing a key vanishingly small.
Additional security can be provided by using `Associated Data' when encrypting a file.
This data must be used to decrypt the data, although it is not part of the encrypted file
\cite{RFC8439}.

\subsection{Message Forgery}
Using Curve25519 and Ed25519 authenticates the content of the encrypted file header.
Using ChaCha20-ietf-Poly1305 authenticates the content of each segment of the encrypted cipher-text. 

\subsection{No File Updates Permitted}
Implementations MUST NOT update encrypted files.
Once written, a section of the file must never be altered.

\section{References}

\begin{thebibliography}{XXXXXXX}

% Normative references

\bibitem[RFC2119]{RFC2119}
  Bradner, S.,\\
  \emph{"Key words for use in RFCs to Indicate Requirement Levels", BCP 14, RFC 2119},\\
  \url{https://www.rfc-editor.org/info/rfc2119},\\
  March 1997.

\bibitem[RFC7748]{RFC7748}
  A. Langley, Google, M. Hamburg, Rambus Cryptography Research, S. Turner, sn3rd,\\
  \emph{"Elliptic Curves for Security", RFC7748},\\
  \url{https://tools.ietf.org/html/rfc7748},\\
  January 2016

\bibitem[RFC8032]{RFC8032}
  S. Josefsson, SJD AB, I. Liusvaara,\\
  \emph{"Edwards-Curve Digital Signature Algorithm (EdDSA)", RFC8032},\\
  \url{https://tools.ietf.org/html/rfc8032},\\
  January 2017

\bibitem[RFC8439]{RFC8439}
  Y. Nir, Dell EMC, A. Langley, Google, Inc.,\\
  \emph{"ChaCha20 and Poly1305 for IETF Protocols", RFC8439},\\
  \url{https://tools.ietf.org/html/rfc8439},\\
  June 2018

% Informational references

\bibitem[RFC5246]{RFC5246}
  Dierks, T., Rescorla, E.,\\
  \emph{"The Transport Layer Security (TLS) Protocol Version 1.2", RFC 5246},\\
  \url{https://www.rfc-editor.org/info/rfc5246},\\
  August 2008.

\end{thebibliography}

\end{document}

\title{GA4GH File Encryption Standard}
\date{\headdate}
\maketitle
\begin{quote}\small
The master version of this document can be found at
\url{https://github.com/samtools/hts-specs}.\\
This printing is version~\commitdesc\ from that repository,
last modified on the date shown above.
\end{quote}
\vspace*{1em}

\begin{abstract}
This document describes the format for Global Alliance for Genomics and Health (GA4GH) encrypted and authenticated
files.
Encryption helps to prevent accidental disclosure of confidential information.
Allowing programs to directly read and write data in an encrypted format reduces the chance of such disclosure.
The format described here can be used to encrypt any underlying file format.
It also allows for seeking on the encrypted data.
In particular indexes on the plain text version can also be used on the encrypted file without modification.
\end{abstract}
\newpage
\tableofcontents
\newpage
\section{Introduction}
\subsection{Purpose}
By its nature, genomic data can include information of a confidential nature about the health of individuals.
It is important that such information is not accidentally disclosed.
One part of the defence against such disclosure is to, as much as possible, keep the data in an encrypted format.

This document describes a file format that can be used to store data in an encrypted and authenticated state.
Existing applications can, with minimal modification, read and write data in the encrypted format.
The choice of encryption also allows the encrypted data to be read starting from any location, facilitating indexed
access to files.

\subsection{Requirements}
The key words ``MUST'', ``MUST NOT'', ``REQUIRED'', ``SHALL'', ``SHALL NOT'', ``SHOULD'', ``SHOULD NOT'',
``RECOMMENDED'', ``MAY'', and ``OPTIONAL'' in this document are to be interpreted as described in \cite{RFC2119}.

\subsection{Terminology}
\begin{description}
\item[Elliptic-curve cryptography (ECC)]
An approach to public-key cryptography based on the algebraic structure of elliptic curves over finite fields.
\item[Curve25519]
A widely used FIPS-140 approved ECC algorithm not encumbered by any patents \cite{RFC7748}. 
\item[Ed25519]
An Edwards-curve Digital Signature Algorithm (EdDSA) over Curve25519 \cite{RFC8032}.
\item[ChaCha20-ietf-Poly1305]
ChaCha20 is a symmetric stream cipher built on a pseudo-random function that gives the advantage that one can 
efficiently seek to any position in the key stream in constant time.
It is not patented.
Poly1305 is a cryptographic message authentication code (MAC).
It can be used to verify the data integrity and the authenticity of a message \cite{RFC8439}.

\item[cipher-text]~\\
The encrypted version of the data.

\item[plain-text]~\\
The unencrypted version of the data.
 
\end{description}

\section{Encrypted Representation Overview}
The encrypted file consists of four parts:

\begin{itemize}
\item An unencrypted header, containing a magic number, version number, length of public key 
(may be zero), public key, a header signature indicator, and the length of the encrypted header.

\item An encrypted header, which is encrypted using an asymmetric encryption algorithm.
It lists the encryption key and nonce needed to decrypt the encrypted data section.

\item A 64-byte Ed25519 signature validating the encrypted header.

\item The encrypted data.
This is the actual application data.
It is encrypted using a symmetric encryption algorithm as described in the encrypted header.
The data is encrypted in 64K segments with a 16 byte MAC appended at the end of each segment.
\end{itemize}

\section{Detailed Specification}
\subsection{Overall Conventions}
\subsubsection{Hexadecimal Numbers}
Hexadecimal values are written using the digits 0-9, and letters a-f for values 10-15.
Values are written with the most-significant digit on the left, and prefixed with "0x".

\subsubsection{Byte Ordering}
The basic data size is the byte (8 bits).
Multi-byte values need to be stored in a defined order.
This order will be one of the following:
\begin{description}
\item[Least-significant byte first (``little-endian'')]~\\
The value 1234 decimal (0x4d2) is stored as the byte stream 0xd2 0x04.

\item[Most-significant byte first (``big-endian'')]~\\
The value 1234 decimal (0x4d2) is stored as the byte stream 0x04 0xd2.
\end{description}

\subsubsection{Integer Types}
Integers can be either signed or unsigned.
Signed values are stored in two's complement form.

\subsubsection{Multi-byte Integer Types}
\begin{center}
\begin{tabular}{l l l l}
\hline
\textbf{Name} & \textbf{Byte Ordering} & \textbf{Integer Type} & \textbf{Size (bytes)} \\
\hline
byte & & unsigned & 1 \\
le\_int32 & little-endian & signed & 4 \\
le\_uint32 & little-endian & unsigned & 4 \\
le\_int64 & little-endian & signed & 8 \\
le\_uint64 & little-endian & unsigned & 8 \\
be\_uint128 & big-endian & unsigned & 16 \\
\end{tabular}
\end{center}

\subsubsection{Vectors}
A vector is a stream of elements of the same type (which may be a structure).
The number of items may be specified either as a constant value or in reference to a known integer value (for
example, a variable previously read from the file).

\begin{verbatim}
le_in32   num     // Number of v2 array elements
le_int32  v1[8]   // Eight four-byte little-endian integers
le_int64  v2[num] // 'num' eight-byte little-endian integers
\end{verbatim}

When vectors are serialized to a file, the elements are written with no padding between them.

\subsubsection{Structures}
Structure types may be defined for convenience.
The syntax for definition is similar to that of C.

\begin{verbatim}
struct demo {
  byte string[8];
  le_int32 number1;
  le_uint64 number2;
};
\end{verbatim}

When structures are serialized to a file, elements are written in the given order with no padding between them.
So the structure above would be written as twenty bytes - eight for the array `string', four for the integer `number1',
and eight for the integer `number2'.

\subsubsection{Enumerated Types}
Enumerated types may only take one of a given set of values.
The data type used to store the enumerated value is given in angle brackets after the type name.
Every element of an enumerated type must be assigned a value.
It is not valid to compare values between two enumerated types.

\begin{verbatim}
enum Animal<le_uint32> {
  cat    = 1;
  dog    = 2;
  rabbit = 3;
};
\end{verbatim}

\subsubsection{Variants}
Parts of structures may vary depending on information available at the time of decoding.
Which variant to use is selected by an enumerated type.
There must be a case for every possible enumerated value.
Cases have limited fall-through.
Consecutive cases with no fields in between all contain the same fields.

\begin{verbatim}
struct AnimalFeatures {
  select (enum Animal) {
    case cat:
    case dog:
      le_uint32 hairyness;
      le_uint32 whisker_length;

    case rabbit:
      le_uint32 ear_length;
  };
};
\end{verbatim}

For the `cat' and `dog' cases, `struct AnimalFeatures' is eight bytes long and contains two unsigned four-byte
little-endian values.
For the `rabbit' case it is four bytes long and contains a single four-byte little-endian value.

If the cases are different lengths (as above), then the size of the overall structure depends on the variant chosen.
There is NO padding to make the cases the same length unless it is explicitly defined.

\subsection{Unencrypted Header}

The file starts with an unencrypted header, with the following structure:

\begin{verbatim}
struct Unencrypted_header {
  byte      magic_number[8];
  le_uint32 version;
  le_uint32 public_key_length;
  byte      public_key[public_key_length];
  le_uint32 header_len;
};
\end{verbatim}

The magic\_number is the ASCII representation of the string ``crypt4gh''.

The version number is stored as a four-byte little-endian unsigned integer.
The current version number is 1.

The public key length may be zero.
If it is greater than zero then it is the length of the public key component of the private key used to sign the
header (this is not the same as the public key used to encrypt the header).

hdr\_len is the sum of the lengths of the unencrypted and encrypted headers.
If a public key is specified then it also includes the 64-byte header signature field appended to the encrypted header.
It is stored as a four-byte little-endian unsigned integer.
As it includes the unencrypted header, hdr\_len will always have a value of at least 16.

The current byte representation of the magic number and version is:
\begin{verbatim}
0x63 0x72 0x79 0x70 0x74 0x34 0x67 0x68 0x01 0x00 0x00 0x00
============= magic_number============= ===== version =====
\end{verbatim}

Possible header configurations:

If no public key is specified:
\begin{verbatim}
[magic number][version][0][header len][encrypted header]
\end{verbatim}

If a public key is specified:
\begin{verbatim}
[magic number][version][pub key len][public key][header len][encrypted header][header signature]
\end{verbatim}

\subsection{Encrypted Header}
\subsubsection{Encryption Method}
The encrypted header is encoded in the Curve25519 message format \cite{RFC7748}.

\subsubsection{Header Signature}
The header signature is generated using the Ed25519 algorithm \cite{RFC8032}.

\subsubsection{Plain-text Format}
The plain-text data encoded in the encrypted header has the following overall structure:

The `Encryption\_parameters' type is defined as:

\begin{verbatim}
enum Encryption_method<le_uint32> {
  chacha20_ietf_poly1305 = 0;
};

struct Encryption_parameters {
  byte	checksum[32];
  enum Encryption_method<le_uint32> method;

  select (method) {
    case chacha20_ietf_poly1305:
      byte       key[32];
      be_uint128 nonce;
  };
};
\end{verbatim}

`checksum' is an optional 32-byte SHA-256 checksum of the (unencrypted) data in the file.

`method' is an enumerated type that describes the type of encryption to be used.

`key' is a secret encryption key.
Treated as a concatenation of eight 32-bit little-endian integers.

`nonce' is a unique initialization vector.
In IETF version it is 12 bytes long and if combined with a 4 byte counter to produce a 16 byte initialization
vector (IV).

\subsection{Encrypted Data}
\subsubsection{ChaCha20 Mode Encryption}

Internally, ChaCha20 works like a block cipher used in counter mode.
It includes an internal block counter to avoid incrementing the nonce after each block.
The cipher-text is the message combined with the output of the stream cipher using the XOR operation, 
and doesn't include any authentication tag.
In IETF mode the nonce is 96 bits long and the counter is 32 bits long.

ChaCha20-Poly1305 uses ChaCha20 to encrypt a message and uses the Poly1305 algorithm to
generate a 16-byte MAC over the encrypted data, which is appended at the end of the 
cipher-text.
The MAC is generated for the whole cipher-text that is provided.
It is not possible to authenticate partially decrypted data.

While it is possible to decrypt individual blocks in ChaCha20, to authenticate the content 
it is necessary to decrypt the entire cipher-text,  with the authentication tag at the end. 
To retain streaming and random access capabilities it is necessary to ensure that segments 
of the data can be authenticated, without having to read and process the whole file.
In this format the cipher-text is divided into 64K segments.
A MAC is generated for each segment individually and appended to each 64K block of cipher-text.

A dual counter method is used, where the nonce is incremented for each 64K segment, and
the counter is incremented within each segment, producing a unique IV (nonce+counter) for 
each cipher block even if the cipher-text is very large. 

The cipher-text is decrypted by authenticating and decrypting the segment(s) enclosing
the requested byte range. 

Implementation details for ChaCha20-ietf-Poly1305are described in \cite{RFC8439}.

\subsubsection{Partial Blocks and Segments}
It is likely that the end of the file will not occur at an exact multiple of the cipher block or 
64K segment length.
In such cases, a block or segment must be padded to enable calculation of the authentication tag. 
Extra care must be taken to prevent a padding oracle attack (\url{https://en.wikipedia.org/wiki/Padding_oracle_attack}).

\section{Security Considerations}
\subsection{Threat Model}
This format is designed to protect files at rest and in transport from accidental disclosure.
Using authenticated encryption in individual segments mirrors solutions like Transport Layer 
Security (TLS) as described in \cite{RFC5246} and prevent undetected modification of segments.

\subsection{Selection of Key and Nonce}
The security of the format depends on attackers not being able to guess the encryption key (and to a lesser extent 
the nonce).
The encryption key MUST be generated using a cryptographically-secure pseudo-random number generator.
This makes the chance of guessing a key vanishingly small.
Additional security can be provided by using `Associated Data' when encrypting a file.
This data must be used to decrypt the data, although it is not part of the encrypted file
\cite{RFC8439}.

\subsection{Message Forgery}
Using Curve25519 and Ed25519 authenticates the content of the encrypted file header.
Using ChaCha20-ietf-Poly1305 authenticates the content of each segment of the encrypted cipher-text. 

\subsection{No File Updates Permitted}
Implementations MUST NOT update encrypted files.
Once written, a section of the file must never be altered.

\section{References}

\begin{thebibliography}{XXXXXXX}

% Normative references

\bibitem[RFC2119]{RFC2119}
  Bradner, S.,\\
  \emph{"Key words for use in RFCs to Indicate Requirement Levels", BCP 14, RFC 2119},\\
  \url{https://www.rfc-editor.org/info/rfc2119},\\
  March 1997.

\bibitem[RFC7748]{RFC7748}
  A. Langley, Google, M. Hamburg, Rambus Cryptography Research, S. Turner, sn3rd,\\
  \emph{"Elliptic Curves for Security", RFC7748},\\
  \url{https://tools.ietf.org/html/rfc7748},\\
  January 2016

\bibitem[RFC8032]{RFC8032}
  S. Josefsson, SJD AB, I. Liusvaara,\\
  \emph{"Edwards-Curve Digital Signature Algorithm (EdDSA)", RFC8032},\\
  \url{https://tools.ietf.org/html/rfc8032},\\
  January 2017

\bibitem[RFC8439]{RFC8439}
  Y. Nir, Dell EMC, A. Langley, Google, Inc.,\\
  \emph{"ChaCha20 and Poly1305 for IETF Protocols", RFC8439},\\
  \url{https://tools.ietf.org/html/rfc8439},\\
  June 2018

% Informational references

\bibitem[RFC5246]{RFC5246}
  Dierks, T., Rescorla, E.,\\
  \emph{"The Transport Layer Security (TLS) Protocol Version 1.2", RFC 5246},\\
  \url{https://www.rfc-editor.org/info/rfc5246},\\
  August 2008.

\end{thebibliography}

\end{document}

\title{GA4GH File Encryption Standard}
\date{\headdate}
\maketitle
\begin{quote}\small
The master version of this document can be found at
\url{https://github.com/samtools/hts-specs}.\\
This printing is version~\commitdesc\ from that repository,
last modified on the date shown above.
\end{quote}
\vspace*{1em}

\begin{abstract}
This document describes the format for Global Alliance for Genomics and Health (GA4GH) encrypted and authenticated
files.
Encryption helps to prevent accidental disclosure of confidential information.
Allowing programs to directly read and write data in an encrypted format reduces the chance of such disclosure.
The format described here can be used to encrypt any underlying file format.
It also allows for seeking on the encrypted data.
In particular indexes on the plain text version can also be used on the encrypted file without modification.
\end{abstract}
\newpage
\tableofcontents
\newpage
\section{Introduction}
\subsection{Purpose}
By its nature, genomic data can include information of a confidential nature about the health of individuals.
It is important that such information is not accidentally disclosed.
One part of the defence against such disclosure is to, as much as possible, keep the data in an encrypted format.

This document describes a file format that can be used to store data in an encrypted and authenticated state.
Existing applications can, with minimal modification, read and write data in the encrypted format.
The choice of encryption also allows the encrypted data to be read starting from any location, facilitating indexed
access to files.

\subsection{Requirements}
The key words ``MUST'', ``MUST NOT'', ``REQUIRED'', ``SHALL'', ``SHALL NOT'', ``SHOULD'', ``SHOULD NOT'',
``RECOMMENDED'', ``MAY'', and ``OPTIONAL'' in this document are to be interpreted as described in \cite{RFC2119}.

\subsection{Terminology}
\begin{description}
\item[Elliptic-curve cryptography (ECC)]
An approach to public-key cryptography based on the algebraic structure of elliptic curves over finite fields.
\item[Curve25519]
A widely used FIPS-140 approved ECC algorithm not encumbered by any patents \cite{RFC7748}. 
\item[Ed25519]
An Edwards-curve Digital Signature Algorithm (EdDSA) over Curve25519 \cite{RFC8032}.
\item[ChaCha20-ietf-Poly1305]
ChaCha20 is a symmetric stream cipher built on a pseudo-random function that gives the advantage that one can 
efficiently seek to any position in the key stream in constant time.
It is not patented.
Poly1305 is a cryptographic message authentication code (MAC).
It can be used to verify the data integrity and the authenticity of a message \cite{RFC8439}.

\item[cipher-text]~\\
The encrypted version of the data.

\item[plain-text]~\\
The unencrypted version of the data.
 
\end{description}

\section{Encrypted Representation Overview}
The encrypted file consists of four parts:

\begin{itemize}
\item An unencrypted header, containing a magic number, version number, length of public key 
(may be zero), public key, a header signature indicator, and the length of the encrypted header.

\item An encrypted header, which is encrypted using an asymmetric encryption algorithm.
It lists the encryption key and nonce needed to decrypt the encrypted data section.

\item A 64-byte Ed25519 signature validating the encrypted header.

\item The encrypted data.
This is the actual application data.
It is encrypted using a symmetric encryption algorithm as described in the encrypted header.
The data is encrypted in 64K segments with a 16 byte MAC appended at the end of each segment.
\end{itemize}

\section{Detailed Specification}
\subsection{Overall Conventions}
\subsubsection{Hexadecimal Numbers}
Hexadecimal values are written using the digits 0-9, and letters a-f for values 10-15.
Values are written with the most-significant digit on the left, and prefixed with "0x".

\subsubsection{Byte Ordering}
The basic data size is the byte (8 bits).
Multi-byte values need to be stored in a defined order.
This order will be one of the following:
\begin{description}
\item[Least-significant byte first (``little-endian'')]~\\
The value 1234 decimal (0x4d2) is stored as the byte stream 0xd2 0x04.

\item[Most-significant byte first (``big-endian'')]~\\
The value 1234 decimal (0x4d2) is stored as the byte stream 0x04 0xd2.
\end{description}

\subsubsection{Integer Types}
Integers can be either signed or unsigned.
Signed values are stored in two's complement form.

\subsubsection{Multi-byte Integer Types}
\begin{center}
\begin{tabular}{l l l l}
\hline
\textbf{Name} & \textbf{Byte Ordering} & \textbf{Integer Type} & \textbf{Size (bytes)} \\
\hline
byte & & unsigned & 1 \\
le\_int32 & little-endian & signed & 4 \\
le\_uint32 & little-endian & unsigned & 4 \\
le\_int64 & little-endian & signed & 8 \\
le\_uint64 & little-endian & unsigned & 8 \\
be\_uint128 & big-endian & unsigned & 16 \\
\end{tabular}
\end{center}

\subsubsection{Vectors}
A vector is a stream of elements of the same type (which may be a structure).
The number of items may be specified either as a constant value or in reference to a known integer value (for
example, a variable previously read from the file).

\begin{verbatim}
le_in32   num     // Number of v2 array elements
le_int32  v1[8]   // Eight four-byte little-endian integers
le_int64  v2[num] // 'num' eight-byte little-endian integers
\end{verbatim}

When vectors are serialized to a file, the elements are written with no padding between them.

\subsubsection{Structures}
Structure types may be defined for convenience.
The syntax for definition is similar to that of C.

\begin{verbatim}
struct demo {
  byte string[8];
  le_int32 number1;
  le_uint64 number2;
};
\end{verbatim}

When structures are serialized to a file, elements are written in the given order with no padding between them.
So the structure above would be written as twenty bytes - eight for the array `string', four for the integer `number1',
and eight for the integer `number2'.

\subsubsection{Enumerated Types}
Enumerated types may only take one of a given set of values.
The data type used to store the enumerated value is given in angle brackets after the type name.
Every element of an enumerated type must be assigned a value.
It is not valid to compare values between two enumerated types.

\begin{verbatim}
enum Animal<le_uint32> {
  cat    = 1;
  dog    = 2;
  rabbit = 3;
};
\end{verbatim}

\subsubsection{Variants}
Parts of structures may vary depending on information available at the time of decoding.
Which variant to use is selected by an enumerated type.
There must be a case for every possible enumerated value.
Cases have limited fall-through.
Consecutive cases with no fields in between all contain the same fields.

\begin{verbatim}
struct AnimalFeatures {
  select (enum Animal) {
    case cat:
    case dog:
      le_uint32 hairyness;
      le_uint32 whisker_length;

    case rabbit:
      le_uint32 ear_length;
  };
};
\end{verbatim}

For the `cat' and `dog' cases, `struct AnimalFeatures' is eight bytes long and contains two unsigned four-byte
little-endian values.
For the `rabbit' case it is four bytes long and contains a single four-byte little-endian value.

If the cases are different lengths (as above), then the size of the overall structure depends on the variant chosen.
There is NO padding to make the cases the same length unless it is explicitly defined.

\subsection{Unencrypted Header}

The file starts with an unencrypted header, with the following structure:

\begin{verbatim}
struct Unencrypted_header {
  byte      magic_number[8];
  le_uint32 version;
  le_uint32 public_key_length;
  byte      public_key[public_key_length];
  le_uint32 header_len;
};
\end{verbatim}

The magic\_number is the ASCII representation of the string ``crypt4gh''.

The version number is stored as a four-byte little-endian unsigned integer.
The current version number is 1.

The public key length may be zero.
If it is greater than zero then it is the length of the public key component of the private key used to sign the
header (this is not the same as the public key used to encrypt the header).

hdr\_len is the sum of the lengths of the unencrypted and encrypted headers.
If a public key is specified then it also includes the 64-byte header signature field appended to the encrypted header.
It is stored as a four-byte little-endian unsigned integer.
As it includes the unencrypted header, hdr\_len will always have a value of at least 16.

The current byte representation of the magic number and version is:
\begin{verbatim}
0x63 0x72 0x79 0x70 0x74 0x34 0x67 0x68 0x01 0x00 0x00 0x00
============= magic_number============= ===== version =====
\end{verbatim}

Possible header configurations:

If no public key is specified:
\begin{verbatim}
[magic number][version][0][header len][encrypted header]
\end{verbatim}

If a public key is specified:
\begin{verbatim}
[magic number][version][pub key len][public key][header len][encrypted header][header signature]
\end{verbatim}

\subsection{Encrypted Header}
\subsubsection{Encryption Method}
The encrypted header is encoded in the Curve25519 message format \cite{RFC7748}.

\subsubsection{Header Signature}
The header signature is generated using the Ed25519 algorithm \cite{RFC8032}.

\subsubsection{Plain-text Format}
The plain-text data encoded in the encrypted header has the following overall structure:

The `Encryption\_parameters' type is defined as:

\begin{verbatim}
enum Encryption_method<le_uint32> {
  chacha20_ietf_poly1305 = 0;
};

struct Encryption_parameters {
  byte	checksum[32];
  enum Encryption_method<le_uint32> method;

  select (method) {
    case chacha20_ietf_poly1305:
      byte       key[32];
      be_uint128 nonce;
  };
};
\end{verbatim}

`checksum' is an optional 32-byte SHA-256 checksum of the (unencrypted) data in the file.

`method' is an enumerated type that describes the type of encryption to be used.

`key' is a secret encryption key.
Treated as a concatenation of eight 32-bit little-endian integers.

`nonce' is a unique initialization vector.
In IETF version it is 12 bytes long and if combined with a 4 byte counter to produce a 16 byte initialization
vector (IV).

\subsection{Encrypted Data}
\subsubsection{ChaCha20 Mode Encryption}

Internally, ChaCha20 works like a block cipher used in counter mode.
It includes an internal block counter to avoid incrementing the nonce after each block.
The cipher-text is the message combined with the output of the stream cipher using the XOR operation, 
and doesn't include any authentication tag.
In IETF mode the nonce is 96 bits long and the counter is 32 bits long.

ChaCha20-Poly1305 uses ChaCha20 to encrypt a message and uses the Poly1305 algorithm to
generate a 16-byte MAC over the encrypted data, which is appended at the end of the 
cipher-text.
The MAC is generated for the whole cipher-text that is provided.
It is not possible to authenticate partially decrypted data.

While it is possible to decrypt individual blocks in ChaCha20, to authenticate the content 
it is necessary to decrypt the entire cipher-text,  with the authentication tag at the end. 
To retain streaming and random access capabilities it is necessary to ensure that segments 
of the data can be authenticated, without having to read and process the whole file.
In this format the cipher-text is divided into 64K segments.
A MAC is generated for each segment individually and appended to each 64K block of cipher-text.

A dual counter method is used, where the nonce is incremented for each 64K segment, and
the counter is incremented within each segment, producing a unique IV (nonce+counter) for 
each cipher block even if the cipher-text is very large. 

The cipher-text is decrypted by authenticating and decrypting the segment(s) enclosing
the requested byte range. 

Implementation details for ChaCha20-ietf-Poly1305are described in \cite{RFC8439}.

\subsubsection{Partial Blocks and Segments}
It is likely that the end of the file will not occur at an exact multiple of the cipher block or 
64K segment length.
In such cases, a block or segment must be padded to enable calculation of the authentication tag. 
Extra care must be taken to prevent a padding oracle attack (\url{https://en.wikipedia.org/wiki/Padding_oracle_attack}).

\section{Security Considerations}
\subsection{Threat Model}
This format is designed to protect files at rest and in transport from accidental disclosure.
Using authenticated encryption in individual segments mirrors solutions like Transport Layer 
Security (TLS) as described in \cite{RFC5246} and prevent undetected modification of segments.

\subsection{Selection of Key and Nonce}
The security of the format depends on attackers not being able to guess the encryption key (and to a lesser extent 
the nonce).
The encryption key MUST be generated using a cryptographically-secure pseudo-random number generator.
This makes the chance of guessing a key vanishingly small.
Additional security can be provided by using `Associated Data' when encrypting a file.
This data must be used to decrypt the data, although it is not part of the encrypted file
\cite{RFC8439}.

\subsection{Message Forgery}
Using Curve25519 and Ed25519 authenticates the content of the encrypted file header.
Using ChaCha20-ietf-Poly1305 authenticates the content of each segment of the encrypted cipher-text. 

\subsection{No File Updates Permitted}
Implementations MUST NOT update encrypted files.
Once written, a section of the file must never be altered.

\section{References}

\begin{thebibliography}{XXXXXXX}

% Normative references

\bibitem[RFC2119]{RFC2119}
  Bradner, S.,\\
  \emph{"Key words for use in RFCs to Indicate Requirement Levels", BCP 14, RFC 2119},\\
  \url{https://www.rfc-editor.org/info/rfc2119},\\
  March 1997.

\bibitem[RFC7748]{RFC7748}
  A. Langley, Google, M. Hamburg, Rambus Cryptography Research, S. Turner, sn3rd,\\
  \emph{"Elliptic Curves for Security", RFC7748},\\
  \url{https://tools.ietf.org/html/rfc7748},\\
  January 2016

\bibitem[RFC8032]{RFC8032}
  S. Josefsson, SJD AB, I. Liusvaara,\\
  \emph{"Edwards-Curve Digital Signature Algorithm (EdDSA)", RFC8032},\\
  \url{https://tools.ietf.org/html/rfc8032},\\
  January 2017

\bibitem[RFC8439]{RFC8439}
  Y. Nir, Dell EMC, A. Langley, Google, Inc.,\\
  \emph{"ChaCha20 and Poly1305 for IETF Protocols", RFC8439},\\
  \url{https://tools.ietf.org/html/rfc8439},\\
  June 2018

% Informational references

\bibitem[RFC5246]{RFC5246}
  Dierks, T., Rescorla, E.,\\
  \emph{"The Transport Layer Security (TLS) Protocol Version 1.2", RFC 5246},\\
  \url{https://www.rfc-editor.org/info/rfc5246},\\
  August 2008.

\end{thebibliography}

\end{document}

\title{GA4GH File Encryption Standard}
\date{\headdate}
\maketitle
\begin{quote}\small
The master version of this document can be found at
\url{https://github.com/samtools/hts-specs}.\\
This printing is version~\commitdesc\ from that repository,
last modified on the date shown above.
\end{quote}
\vspace*{1em}

\begin{abstract}
This document describes the format for Global Alliance for Genomics and Health (GA4GH) encrypted and authenticated
files.
Encryption helps to prevent accidental disclosure of confidential information.
Allowing programs to directly read and write data in an encrypted format reduces the chance of such disclosure.
The format described here can be used to encrypt any underlying file format.
It also allows for seeking on the encrypted data.
In particular indexes on the plain text version can also be used on the encrypted file without modification.
\end{abstract}
\newpage
\tableofcontents
\newpage
\section{Introduction}
\subsection{Purpose}
By its nature, genomic data can include information of a confidential nature about the health of individuals.
It is important that such information is not accidentally disclosed.
One part of the defence against such disclosure is to, as much as possible, keep the data in an encrypted format.

This document describes a file format that can be used to store data in an encrypted and authenticated state.
Existing applications can, with minimal modification, read and write data in the encrypted format.
The choice of encryption also allows the encrypted data to be read starting from any location, facilitating indexed
access to files.

\subsection{Requirements}
The key words ``MUST'', ``MUST NOT'', ``REQUIRED'', ``SHALL'', ``SHALL NOT'', ``SHOULD'', ``SHOULD NOT'',
``RECOMMENDED'', ``MAY'', and ``OPTIONAL'' in this document are to be interpreted as described in \cite{RFC2119}.

\subsection{Terminology}
\begin{description}
\item[Elliptic-curve cryptography (ECC)]
An approach to public-key cryptography based on the algebraic structure of elliptic curves over finite fields.
\item[Curve25519]
A widely used FIPS-140 approved ECC algorithm not encumbered by any patents \cite{RFC7748}.
\item[Ed25519]
An Edwards-curve Digital Signature Algorithm (EdDSA) over Curve25519 \cite{RFC8032}.
\item[ChaCha20-ietf-Poly1305]
ChaCha20 is a symmetric stream cipher built on a pseudo-random function that gives the advantage that one can
efficiently seek to any position in the key stream in constant time.
It is not patented.
Poly1305 is a cryptographic message authentication code (MAC).
It can be used to verify the data integrity and the authenticity of a message \cite{RFC8439}.

\item[cipher-text]~\\
The encrypted version of the data.

\item[plain-text]~\\
The unencrypted version of the data.

\end{description}

\section{Encrypted Representation Overview}
The encrypted file consists of four parts:

\begin{itemize}
\item An unencrypted header, containing a magic number, version number, length of public key
(may be zero), public key, a header signature indicator, and the length of the encrypted header.

\item An encrypted header, which is encrypted using an asymmetric encryption algorithm.
It lists the encryption key and nonce needed to decrypt the encrypted data section.

\item A 64-byte Ed25519 signature validating the encrypted header.

\item The encrypted data.
This is the actual application data.
It is encrypted using a symmetric encryption algorithm as described in the encrypted header.
The data is encrypted in 64K segments with a 16 byte MAC appended at the end of each segment.
\end{itemize}

\section{Detailed Specification}
\subsection{Overall Conventions}
\subsubsection{Hexadecimal Numbers}
Hexadecimal values are written using the digits 0-9, and letters a-f for values 10-15.
Values are written with the most-significant digit on the left, and prefixed with "0x".

\subsubsection{Byte Ordering}
The basic data size is the byte (8 bits).
All multi-byte vales are stored in least-significant byte first (``little-endian'') order.
For example, the value 1234 decimal (0x4d2) is stored as the byte stream 0xd2 0x04.

\subsubsection{Integer Types}
Integers can be either signed or unsigned.
Signed values are stored in two's complement form.

\subsubsection{Multi-byte Integer Types}
\begin{center}
\begin{tabular}{l l l l}
\hline
\textbf{Name} & \textbf{Byte Ordering} & \textbf{Integer Type} & \textbf{Size (bytes)} \\
\hline
byte & & unsigned & 1 \\
le\_int32 & little-endian & signed & 4 \\
le\_uint32 & little-endian & unsigned & 4 \\
le\_int64 & little-endian & signed & 8 \\
le\_uint64 & little-endian & unsigned & 8 \\
le\_uint96 & little-endian & unsigned & 12 \\
\end{tabular}
\end{center}

\subsubsection{Vectors}
A vector is a stream of elements of the same type (which may be a structure).
The number of items may be specified either as a constant value or in reference to a known integer value (for
example, a variable previously read from the file).

\begin{verbatim}
le_in32   num     // Number of v2 array elements
le_int32  v1[8]   // Eight four-byte little-endian integers
le_int64  v2[num] // 'num' eight-byte little-endian integers
\end{verbatim}

When vectors are serialized to a file, the elements are written with no padding between them.

\subsubsection{Structures}
Structure types may be defined for convenience.
The syntax for definition is similar to that of C.

\begin{verbatim}
struct demo {
  byte string[8];
  le_int32 number1;
  le_uint64 number2;
};
\end{verbatim}

When structures are serialized to a file, elements are written in the given order with no padding between them.
So the structure above would be written as twenty bytes - eight for the array `string', four for the integer `number1',
and eight for the integer `number2'.

\subsubsection{Enumerated Types}
Enumerated types may only take one of a given set of values.
The data type used to store the enumerated value is given in angle brackets after the type name.
Every element of an enumerated type must be assigned a value.
It is not valid to compare values between two enumerated types.

\begin{verbatim}
enum Animal<le_uint32> {
  cat    = 1;
  dog    = 2;
  rabbit = 3;
};
\end{verbatim}

\subsubsection{Variants}
Parts of structures may vary depending on information available at the time of decoding.
Which variant to use is selected by an enumerated type.
There must be a case for every possible enumerated value.
Cases have limited fall-through.
Consecutive cases with no fields in between all contain the same fields.

\begin{verbatim}
struct AnimalFeatures {
  select (enum Animal) {
    case cat:
    case dog:
      le_uint32 hairyness;
      le_uint32 whisker_length;

    case rabbit:
      le_uint32 ear_length;
  };
};
\end{verbatim}

For the `cat' and `dog' cases, `struct AnimalFeatures' is eight bytes long and contains two unsigned four-byte
little-endian values.
For the `rabbit' case it is four bytes long and contains a single four-byte little-endian value.

If the cases are different lengths (as above), then the size of the overall structure depends on the variant chosen.
There is NO padding to make the cases the same length unless it is explicitly defined.

\subsection{Unencrypted Header}

The file starts with an unencrypted header, with the following structure:

\begin{verbatim}
struct Unencrypted_header {
  byte      magic_number[8];
  le_uint32 version;
  le_uint32 public_key_length;
  byte      public_key[public_key_length];
  le_uint32 header_len;
};
\end{verbatim}

The magic\_number is the ASCII representation of the string ``crypt4gh''.

The version number is stored as a four-byte little-endian unsigned integer.
The current version number is 1.

The public key length may be zero.
If it is greater than zero then it is the length of the public key component of the private key used to sign the
header (this is not the same as the public key used to encrypt the header).

header\_len is the sum of the lengths of the unencrypted and encrypted headers.
If a public key is specified then it also includes the 64-byte header signature field appended to the encrypted header.
It is stored as a four-byte little-endian unsigned integer.
As it includes the unencrypted header, header\_len will always have a value of at least 16.

The current byte representation of the magic number and version is:
\begin{verbatim}
0x63 0x72 0x79 0x70 0x74 0x34 0x67 0x68 0x01 0x00 0x00 0x00
============= magic_number============= ===== version =====
\end{verbatim}

Possible header configurations:

If no public key is specified:
\begin{verbatim}
[magic number][version][0][header len][encrypted header]
\end{verbatim}

If a public key is specified:
\begin{verbatim}
[magic number][version][pub key len][public key][header len][encrypted header][header signature]
\end{verbatim}

\subsection{Encrypted Header}
\subsubsection{Encryption Method}
The encrypted header is encoded in the Curve25519 message format \cite{RFC7748}.

\subsubsection{Header Signature}
The header signature is generated using the Ed25519 algorithm \cite{RFC8032}.

\subsubsection{Plain-text Format}
The plain-text data encoded in the encrypted header has the following overall structure:

The `Encryption\_parameters' type is defined as:

\begin{verbatim}
enum Encryption_method<le_uint32> {
  chacha20_ietf_poly1305 = 0;
};

struct Encryption_parameters {
  byte	checksum[32];
  enum Encryption_method<le_uint32> method;

  select (method) {
    case chacha20_ietf_poly1305:
      byte      key[32];
      le_uint96 nonce;
  };
};
\end{verbatim}

`checksum' is an optional 32-byte SHA-256 checksum of the (unencrypted) data in the file.

`method' is an enumerated type that describes the type of encryption to be used.

`key' is a secret encryption key.
Treated as a concatenation of eight 32-bit little-endian integers.

`nonce' is a unique initialization vector.
In chacha20\_ietf\_poly1305 it is 12 bytes long.

\subsection{Encrypted Data}
\subsubsection{ChaCha20 Mode Encryption}

Internally, ChaCha20 works like a block cipher used in counter mode.
It includes an internal block counter to avoid incrementing the nonce after each block.
The cipher-text is the message combined with the output of the stream cipher using the XOR operation,
and doesn't include any authentication tag.
In IETF mode the nonce is 96 bits long and the counter is 32 bits long.

ChaCha20-Poly1305 uses ChaCha20 to encrypt a message and uses the Poly1305 algorithm to
generate a 16-byte MAC over the encrypted data, which is appended at the end of the
cipher-text.
The MAC is generated for the whole cipher-text that is provided.
It is not possible to authenticate partially decrypted data.

While it is possible to decrypt individual blocks in ChaCha20, to authenticate the content
it is necessary to decrypt the entire cipher-text,  with the authentication tag at the end.
To retain streaming and random access capabilities it is necessary to ensure that segments
of the data can be authenticated, without having to read and process the whole file.
In this format the plain-text is divided into 64KiB segments and encrypted to 64KiB of cipher-text.
A MAC is generated for each encrypted segment individually and appended to each 64KiB block of cipher-text.
This expands the data by 16 bytes, so a 65536 byte plain-text input will become a 65552 byte encrypted
and authenticated output.

A dual counter method is used, where the nonce is incremented after each 64K segment, and
the counter is incremented within each segment, producing a unique IV (nonce+counter) for
each cipher block even if the input is very large.
The initial value used for the counter when encrypting each 64KiB block is 1.

The cipher-text is decrypted by authenticating and decrypting the segment(s) enclosing
the requested byte range.
To decrypt starting from a plain-text position P, first the location of the segment containing that position
must be found.

\begin{verbatim}
seg_start = header_len + floor(P/65536) * 65552
\end{verbatim}

The 65552 bytes (possibly fewer at the end of the file) starting at this position are read to obtain the encrypted
data and 16 byte MAC.
Next the nonce for the segment is calculated.

\begin{verbatim}
seg_nonce = nonce + floor(P/65536)
\end{verbatim}

The key, nonce, encrypted data and MAC are then used to obtain the plain text.
Finally (P mod 65536) bytes at the start are discarded from the plain text to give the desired output.

Implementation details for ChaCha20-ietf-Poly1305are described in \cite{RFC8439}.

\subsubsection{Partial Segments}
It is likely that the end of the file will not occur at an exact multiple of the cipher block or
64K segment length.
In such cases the short block will be encrypted in the way described above, apart from the block size will
now be the number of remaining bytes.
This will produce an encrypted output which will be exactly 16 bytes longer than the input.

If the last block is found to be shorter than 65552 bytes on reading, the last 16 bytes should be taken as
the MAC, and the remaining bytes as the short block to decrypt.
If the last block is found to be shorter than 16 bytes the file is corrupt and an error should be reported.

\section{Security Considerations}
\subsection{Threat Model}
This format is designed to protect files at rest and in transport from accidental disclosure.
Using authenticated encryption in individual segments mirrors solutions like Transport Layer
Security (TLS) as described in \cite{RFC5246} and prevent undetected modification of segments.

\subsection{Selection of Key and Nonce}
The security of the format depends on attackers not being able to guess the encryption key (and to a lesser extent
the nonce).
The encryption key and nonce MUST be generated using a cryptographically-secure pseudo-random number generator.
This makes the chance of guessing a key vanishingly small.
Additional security can be provided by using `Associated Data' when encrypting a file.
This data must be used to decrypt the data, although it is not part of the encrypted file
\cite{RFC8439}.

\subsection{Message Forgery}
Using Curve25519 and Ed25519 authenticates the content of the encrypted file header.
Using ChaCha20-ietf-Poly1305 authenticates the content of each segment of the encrypted cipher-text.

\subsection{No File Updates Permitted}
Implementations MUST NOT update encrypted files.
Once written, a section of the file must never be altered.

\section{References}

\begin{thebibliography}{XXXXXXX}

% Normative references

\bibitem[RFC2119]{RFC2119}
  Bradner, S.,\\
  \emph{"Key words for use in RFCs to Indicate Requirement Levels", BCP 14, RFC 2119},\\
  \url{https://www.rfc-editor.org/info/rfc2119},\\
  March 1997.

\bibitem[RFC7748]{RFC7748}
  A. Langley, Google, M. Hamburg, Rambus Cryptography Research, S. Turner, sn3rd,\\
  \emph{"Elliptic Curves for Security", RFC7748},\\
  \url{https://tools.ietf.org/html/rfc7748},\\
  January 2016

\bibitem[RFC8032]{RFC8032}
  S. Josefsson, SJD AB, I. Liusvaara,\\
  \emph{"Edwards-Curve Digital Signature Algorithm (EdDSA)", RFC8032},\\
  \url{https://tools.ietf.org/html/rfc8032},\\
  January 2017

\bibitem[RFC8439]{RFC8439}
  Y. Nir, Dell EMC, A. Langley, Google, Inc.,\\
  \emph{"ChaCha20 and Poly1305 for IETF Protocols", RFC8439},\\
  \url{https://tools.ietf.org/html/rfc8439},\\
  June 2018

% Informational references

\bibitem[RFC5246]{RFC5246}
  Dierks, T., Rescorla, E.,\\
  \emph{"The Transport Layer Security (TLS) Protocol Version 1.2", RFC 5246},\\
  \url{https://www.rfc-editor.org/info/rfc5246},\\
  August 2008.

\end{thebibliography}

\end{document}
